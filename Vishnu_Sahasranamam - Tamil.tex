\documentclass[10pt]{article}
\usepackage[utf8]{inputenc}

% Mult—lingual utility package
\usepackage{polyglossia}
\setdefaultlanguage{english}
\setotherlanguage{tamil}
\defaultfontfeatures{Scale=MatchLowercase}
\newfontfamily\tamilfont{Latha}[Script=Tamil]

% Custom utility package to deal with Sanskrit accents in Tamil
\usepackage{stackmacros}

\begin{document} \texttamil{

\heading{\B{\Huge விஷ்ணு ஸஹஸ்ரநாமம் \\ \eng{\huge (Vishnu Sahasranamam in Tamil)}}}
\newcounter{mycounter}
% śuklāṁbaradharaṁ viṣṇuṁ śaśivarṇaṁ caturbhujam |
% prasannavadanaṁ dhyāyet sarvavighnōpaśāṁtaye || 1 ||
\SlokaLarge
{ஷுக்லா அம்\y{ப}{3}ர\y{த}{4}ர\BU{ம்} விஷ்ணும்}
{ஶஶி \BU{வர்}ணம் சதுர் \y{பு}{4}ஜம்}
{ப்ரஸன்ன வ\y{த}{3}ன\BU{ம்} \y{த்}{4}யாயேத்}
{\BU{ஸ}ர்வ \BU{வி}\y{க்}{4}னோப ஷாந்தயே}
\phantom{\countr}
\BU{\huge பூர்வ பீடிகா \eng{\huge (Poorva Peetika)}}

% vyāsaṁ vasiṣṭhanaptāraṁ śakteḥ pautramakalmaṣam |
% parāśarātmajaṁ vaṁde śukatātaṁ tapōnidhim || 3 ||
\SlokaLarge
{வ்யாஸம் வஸிஷ்ட \BU{ந}ப்தாரம்}
{ஶக்தெ\B{:} பௌத்ரம் — அகல்மஷம்}
{பராஷர — ஆத்மஜ\BU{ம்} வந்\y{தே}{3}}
{ஷுக \BU{தா}தம் தபோ நி\y{தி}{4}ம்}

% vyāsāya viṣṇurūpāya vyāsarūpāya viṣṇave |
% namō vai brahmanidhaye vāsiṣṭhāya namō namaḥ || 4 ||
\SlokaLarge
{வ்யாஸாய விஷ்ணு \BU{ரூ}பாய}
{\BU{வ்யா}ஸ \BU{ரூ}பாய விஷ்ணவே}
{நமோ வை ப்ரஹ்ம \BU{நி}\y{த}{4}யே}
{\BU{வா}ஸிஷ்\BU{\y{டா}{2}}ய நமோ நம\B{:}}

% avikārāya śuddhāya nityāya paramātmane |
% sadaikarūparūpāya viṣṇave sarvajiṣṇave || 5 ||
\SlokaLarge
{அவிகாராய \BU{ஷு\y{த்}{3}}\y{தா}{4}ய}
{\BU{நி}த்யாய பரமாத்மனே}
{ஸ\y{தை}{3}க ரூப \BU{ரூ}பாய}
{\BU{வி}ஷ்ணவே ஸர்வ ஜிஷ்ணவே}

% yasya smaraṇamātreṇa janmasaṁsārabaṁdhanāt |
% vimucyate namastasmai viṣṇave prabhaviṣṇave || 6 ||
\SlokaLarge
{யஸ்ய ஸ்மரண \BU{மா}த்ரேண}
{\BU{ஜ}ன்ம \BU{ஸ}ம்ஸார \y{ப}{3}ந்\y{த}{4}நாத்}
{விமுச்யதே ந\BU{ம:} — தஸ்மை (ந\BU{மஸ்} / தஸ்மை)}
{\BU{வி}ஷ்ணவே ப்ர\y{ப}{4} விஷ்ணவே}

% ōṁ namō viṣṇave prabhaviṣṇave ||
\SlokaStrayLarge
{\B{ஓம்} நமோ விஷ்ணவே ப்ர\y{ப}{4} விஷ்ணவே}

% śrī vaiśaṁpāyana uvāca
% śrutvā dharmānaśeṣeṇa pāvanāni ca sarvaśaḥ |
% yudhiṣṭhiraḥ śāṁtanavaṁ punarevābhyabhyāṣata || 7 ||
\SlokaHeaderLarge
{\B{ஶ்ரீ} வைஷம் பாயந உவாச}
{ஶ்ருத்வா \y{த}{4}ர்மாண் — அ\BU{ஷே}ஶேண} {\BU{பா}வ\BU{னா}னி ச ஸர்வஷ:}
{யு\y{தி}{4}ஶ்டிர: ஷாந்தனவம்} {புனரேவ — அ\y{பி}{4} — அ\y{பா}{4}ஶத (அ\y{ப்}{4}ய / \y{பா}{4}ஶத)}

\newpage
% śrī yudhiṣṭhira uvāca
% kimekaṁ daivataṁ lōke kiṁ vāpyekaṁ parāyaṇaṁ |
% stuvaṁtaḥ kaṁ kamarcaṁtaḥ prāpnuyurmānavāḥ śubham || 8 ||
\SlokaHeaderLarge
{யு\y{தி}{4}ஶ்டிர உவாச}
{கிம் — ஏகம் \y{தை}{3}வ\BU{தம்} லோகே} {\BU{கி}ம் வாபி — \BU{ஏ}கம் (வாப்\BU{யே} / கம்) பராயணம்}
{ஸ்துவந்த: கம் கம் \dash அர்சந்த:} {\BU{ப்ரா}ப்னுயு: — மானவ: (ப்ராப்னுயுர்/மானவா) ஷுபம்}

% kō dharmaḥ sarvadharmāṇāṁ bhavataḥ paramō mataḥ |
% kiṁ japanmucyate jaṁturjanmasaṁsārabaṁdhanāt || 9 ||
\SlokaLarge
{கோ \y{த}{4}ர்ம: ஸர்வ \Y{த}{4}ர்மாணாம்}{\Y{ப}{4}வத: பரமோ மத:}
{கிம் ஜபன் — மமுச்ய\BU{தே} ஜந்து: (ஜந்துர்)}{\BU{ஜ}ன்ம \BU{ஸ}ம்ஸார \y{ப}{3}ந்\y{த}{4}னாத்}

% śrī bhīṣma uvāca
% jagatprabhuṁ devadevamanaṁtaṁ puruṣōttamam |
% stuvannāmasahasreṇa puruṣaḥ satatōtthitaḥ || 10 ||
\SlokaHeaderLarge
{ஶ்ரீ \y{பீ}{4}ஷ்ம உவாச}
{ஜகத் ப்ர\y{பு}{4}ம் \y{தே}{3}வ \y{தே}{3}வம்}{அநந்தம் புருஷோத்தமம்}
{ஸ்துவன் — நாம ஸஹஸ்ரேண}{புருஷ: ஸததோத்\y{தி}{2}த: (ஸததோ / \y{தி}{2}த:)}

\newpage
% tameva cārcayannityaṁ bhaktyā puruṣamavyayam |
% dhyāyan stuvannamasyaṁśca yajamānastameva ca || 11 ||
\SlokaLarge
{தம் — ஏவ (தமேவ) சார்சயன் — நித்யம்} {\y{ப}{4}க்த்யா புருஷம் — அவ்யயம்}
{\y{த்}{4}யாயன் ஸ்துவன் நமஸ்சன் — ஷ்ச} {யஜமான: தம்— ஏவ (யஜமானஸ்/தமேவ) ச}

% anādi nidhanaṁ viṣṇuṁ sarvalōkamaheśvaram |
% lōkādhyakṣaṁ stuvannityaṁ sarvaduḥkhātigō bhavet || 12 ||
\SlokaLarge
{அனா\y{தி}{3} நி\y{த}{4}னம் விஷ்ணும்} {ஸர்வ லோக மஹேஷ்வரம்}
{லோகா — அ\y{த்}{4}யக்ஷம் ஸ்துவன் — நித்யம்} {சர்வ \y{து}{3}க்காதி\y{கொ}{3} \y{ப}{4}வேத்}

% brahmaṇyaṁ sarvadharmajñaṁ lōkānāṁ kīrtivardhanam |
% lōkanāthaṁ mahadbhūtaṁ sarvabhūtabhavōdhbhavam || 13 ||
\SlokaLarge
{\y{ப்}{3}ரஹ்மண்யம் ஸர்வ \Y{த}{4}ர்மண்யம்} {\BU{லோ}கானாம் கீர்தி வர்\y{த}{4}னம்}
{லோகநா\y{த}{2}ம் ம\BU{ஹ\y{த்}{3}} — \y{பூ}{4}தம்} {\BU{ச}ர்வ \Y{பூ}{4}த \y{ப}{4}வோ\y{த்}{3}\y{ப}{4}வம்}

% eṣa me sarvadharmāṇāṁ dharmōdhikatamō mataḥ
% yadbhaktyā puṁḍarīkākṣaṁ stavairarcennaraḥ sadā || 14 ||
\SlokaLarge
{ஏஶ மே ஸர்வ \Y{த}{4}ர்மாணாம்}{\Y{த}{4}ர்மோ — அ\y{தி}{4}கதமோ மத:}
{ய\y{த்}{3} \dash \y{ப}{4}க்த்யா புண்\y{ட}{3}\BU{ரீ}காக்ஷம்}{\BU{ஸ்த}வை அர்சேன் நர: (ஸ்தவைர்/அர்சேன்னர:) ச\y{தா}{3}}

% paramaṁ yō mahattejaḥ paramaṁ yō mahattapaḥ |
% paramaṁ yō mahadbrahma paramaṁ yaḥ parāyaṇam || 15 ||
\SlokaLarge
{பரமம் யோ மஹத்தேஜ:}{பரமம் யோ மஹத்தப:}
{பரமம் யோ மஹத்ப்ரஹ்ம}{பரமம் ய: பராயணம்}

\SlokaLarge
{பவித்ராணாம் பவித்ரம் யோ}{மங்\y{க}{3}ளானாம் ச மங்\y{க}{3}ளம்}
{\y{தை}{3}ய்வதம் \y{தே}{3}வதானாம் ச}{\y{பூ}{4}தானாம் யொவ்யயஹ் பிதா}

\SlokaLarge
{யத ஸர்வாணி \y{பூ}{4}தானி}{\y{ப}{4}வன்த்யாy{தி}{3} யு\y{கா}{3}\y{க}{3}மே}
{யஸ்மின்ஶ்ச ப்ரளயம் யாந்தி}{புனரேவ யு\y{க}{3}க்ஷயே}

\SlokaLarge
{தஸ்ய லோக ப்ர\y{தா}{3}னஸ்ய}{ஜ\y{க}{3}ந்நா\y{த}{2}ஸ்ய \y{பூ}{4}பதே}
{விஷ்னோ: \dash நாம (விஷ்னோர் / நாம) ஸஹஸ்ரம் மே}{ஷ்ருணு பாப \y{ப}{4}யாபஹம்}

\newpage
\SlokaLarge
{யானி நாமானி \y{கெள}{3}ணானி}{வி\y{க்}{2}யாதானி மஹாத்மன:}
{ருஷி\y{பி}{4}: பரி\y{கீ}{3}தானி}{தானி வக்ஷ்யாமி \y{பூ}{4}தயே}

\SlokaLarge
{ருஷிர்‌னாம்‌ நாம் ஸஹஸ்ரஸ்ய}{வேதவ்யாஸோ மஹாமுனி:}
{\y{ச}{2}ந்தோ \dash அனுஷ்டுப்‌ த\y{தா}{2} \y{தே}{3}வோ} {\y{ப}{4}\y{க}{3}வான்‌ \y{தே}{3}வகீ ஸுத:}

% \SlokaLarge
% {அம்ருதாம்சூ’த்பவோ பீஜம்‌ ச’க்திர்தேவகிநந்தன:}
% {த்ரிஸாமா ஹ்ருதயம்‌ தஸ்ய சா’ந்த்யர்த்தே விநியுஜ்யதே}

% \SlokaLarge
% {விஷ்ணும்‌ ஜிஷ்ணும்‌ மஹாவிஷ்ணும்‌ ப்ரபவிஷ்ணும்‌ மஹேச்’வரம்‌}
% {அநேகரூப தைத்யாந்தம்‌ நமாமி புருஷோத்தமம்‌}

% {\SlokaHeaderLarge}
% {ஓம்அஸ்ய ஸ்ரீ விஷ்ணோர்‌}
% {திவ்ய ஸஹஸ்ரநாம ஸ்தோத்ர மஹாமந்த்ரஸ்ய}
% {ஸ்ரீ வேத வ்யாஸோ பகவான்‌ ருஷி:}
% {அனுஷ்டுப்ச்சந்த: ஸ்ரீ மஹாவிஷ்ணு:}
% {பரமாத்மா ஸ்ரீமந்‌ நாராயணோ தேவதா}
% {அம்ருதாம்சூ’த்பவோ பானுரிதி பீஜம்‌}
% {தேவகீ நந்தன: ஸ்ரஷ்டேதி ச’க்தி:}
% {உத்பவ:க்ஷோபணோதேவ இதிபரமோ மந்த்ர:}
% {ச’ங்கப்ருந்‌ நந்தகீ சக்ரீதி கீலகம்‌}
% {சா’ர்ங்கதன்வா கதாதர இத்யஸ்த்ரம்‌}
% {ரதாங்கபாணி—ரக்ஷோப்ய இதிநேத்ரம்‌}
% {த்ரிஸாமா ஸாமக:ஸாமேதி கவசம்‌}
% {ஆனந்தம்‌ பரப்ரஹ்மேதி யோனி:}
% {ருது: ஸுதர்ச’ன : கால இதி திக்பந்த:}
% {ஸ்ரீவிச்’வரூப இதித்யானம்}
% {ஸ்ரீமஹாவிஷ்ணு ப்ரீத்யர்த்தே}
% {ஸ்ரீஸஹஸ்ரநாம ஜபே விநியோக: //}

% \SlokaLarge
% {த்யானம்‌}
% {க்ஷீரோதன்வத்‌ ப்ரதேசே’ சு’சிமணி விலஸத்}
% {ஸைகதேர் மெளக்திகானாம்‌}
% {மாலாக்லுப்தா ஸனஸ்த: ஸ்ஃபடிகமணி}
% {நிபைர்‌ மெளக்திகைர்‌ மண்டிதாங்க:}

% \SlokaLarge
% {சு’ப்ரை—ரப்ரை—ரதப்ரை—ருபரிவிரசிதைர்‌}
% {முக்த பீயூஷ வர்ஷை:}
% {ஆனந்தீ ந: புனீயா தரிநளின கதா}
% {ச’ங்கபாணிர்‌ முகுந்த:}

% \SlokaLarge
% {பூ: பாதெள யஸ்ய நாபிர்‌வியதஸூர நிலச்’:}
% {சந்த்ர ஸூர்யெள ச நேத்ரே}
% {கர்ணாவாசா’ சி’ரோத்யெளர்‌ முகமபி}
% {தஹனோ யஸ்ய வாஸ்தேய மப்தி}
% {அந்தஸ்த்தம்‌ யஸ்ய விச்’வம்‌ ஸுர நர௧௧கோ போகி கந்தர்வ தைத்யை:}
% {சித்ரம் ‌ரம் ரம்யதே தம்‌ த்ரிபுவன வபுஷம்‌ விஷ்ணுமீச’ம்‌ நமாமி}

% \SlokaLarge
% {சா’ந்தாகாரம்‌ புஜகச’யனம்‌}
% {பத்மநாபம்‌ ஸுரேச’ம்‌}
% {விச்’வாதாரம்‌ ௧௧னஸத்ருச’ம்‌}
% {மேகவர்ணம்‌ சு’பாங்கம்‌}
% {லக்ஷ்மீகாந்தம்‌ கமலநயனம்}
% {யோகிஹ்ருத்—த்யானகம்யம்‌}
% {வந்தே விஷ்ணும்‌ பவபயஹரம்‌}
% {ஸர்வலோகைகநாதம்‌}

% \SlokaLarge
% {மேகச்’யாமம்‌ பீதகெளசே’யவாஸம்‌}
% {ஸ்ரீவத்ஸாங்கம்‌ கெளஸ்துபோத்பாஸிதாங்கம்‌}
% {புண்யோபேதம்‌ புண்டரீகாயதாக்ஷம்‌}
% {விஷ்ணும்‌ வந்தே ஸ்ர்வலோகைகநாதம்‌}

% \SlokaLarge
% {நம : ஸமஸ்த பூதானாம்‌}
% {ஆதிபூதாய பூப்ருதே}
% {அனேகரூபரூபாய}
% {விஷ்ணவே ப்ரபவிஷ்ணவே}

% \SlokaLarge
% {ஸச’ங்கசக்ரம்‌ ஸகிரீடகுண்டலம்‌}
% {ஸபீதவஸ்த்ரம்‌ ஸரஸீருஹேக்ஷணம்‌ /}
% {ஸஹாரவக்ஷஸ்த்தல சோ’பிகெளஸ்துபம்‌}
% {நமாமி விஷ்ணும்‌ சி’ரஸா சதுர்ப்புஜம்‌}

% \SlokaLarge
% {சாயாயாம்‌ பாரிஜாதஸ்ய}
% {ஹேம ஸிம்ஹாஸனோபரி}
% {ஆஸீனமம்புத ச்’யாமம்}
% {ஆயதாக்ஷமலங்க்ருதம்‌}

% \SlokaLarge
% {சந்த்ரானனம்‌ சதுர்பாஹும்‌}
% {ஸ்ரீவத்ஸாங்கித வக்ஷஸம்‌}
% {ருக்மிணீ—ஸத்யபாமாப்யாம்‌}
% {ஸஹிதம்‌ க்ருஷ்ணமாச்’ரயே}
% {ஒம்‌ விஸ்வஸ்மை நம}
% {விச்’வம்‌ விஷ்ணுர்‌—வஷட்காரோ}
% {பூத பவ்ய பவத்‌ ப்ரபு:}
% {பூதக்ருத்‌ பூதப்ருத்‌ பாவோ}
% {பூதாத்மா பூதபாவன:}

% \SlokaLarge
% {பூதாத்மா பரமாத்மாச}
% {முக்தானாம்‌ பரமாகதி:}
% {அவ்யய: புருஷ:‌ ஸாக்ஷீ}
% {க்ஷேத்ரஜ்ஞோ(அ)க்ஷர ஏவ ச}

% \SlokaLarge
% {யோகோ யோக விதாம்‌ நேதா}
% {ப்ரதானபுருஷேச்’வர:}
% {நாரஸிம்ஹவபு: ஸ்ரீமான்}
% {கேசவ:புருஷோத்தம:}

% \SlokaLarge
% {ஸர்வ: ச’ர்வ: சி’வ: ஸ்தாணுர்‌}
% {பூதாதிர்‌ நிதிரவ்யய:}
% {ஸம்பவோ பாவனோ பர்த்தா}
% {ப்ரபவ: ப்ரபுரீச்’வர: //4}

% {ஸ்வயம்பூச்‌ ச’ம்பு—ராதித்ய:}
% {புஷ்கராக்ஷோ மஹாஸ்வன:}
% {அநாதி நிதனோ தாதா}
% {விதாதா தாது ருத்தம:}

% \SlokaLarge
% {அப்ரமேயோ ஹ்ருஷீகேச’:}
% {பத்மநாபோ(அ)மரப்ரபு:}
% {விச்’வகர்மா மனுஸ்‌ த்வஷ்டா}
% {ஸ்தவிஷ்ட: ஸ்த்தவிரோ த்ருவ:}

% \SlokaLarge
% {அக்ராஹ்ய: சா’ச்வத: க்ருஷ்ணோ}
% {லோஹிதாக்ஷ: ப்ரதர்த்தன: /}
% {ப்ரபூதஸ்‌ த்ரிககுப்தாம}
% {பவித்ரம்‌ மங்களம்‌ பரம்‌}

% \SlokaLarge
% {ஈசா’ன: ப்ராணத: ப்ராணோ}
% {ஜ்யேஷ்ட்ட: ச்’ரேஷ்ட்ட: ப்ரஜாபதி:}
% {ஹிரண்யகர்ப்போ பூகர்ப்போ}
% {மாதவோ மதுஸூதன:}

% \SlokaLarge
% {ஈச்’வரோ விக்ரமீ தன்வீ}
% {மேதாவீவிக்ரம: க்ரம:}
% {அனுத்தமோ துராதர்ஷ:}
% {க்ருதஜ்ஞ: க்ருதிராத்மவான்}

% \SlokaLarge
% {ஸுரேச’:ச’ரணம்‌ சர்ம}
% {விச்’வரேதா: ப்ரஜாபவ:}
% {அஹ:‌ ஸம்வத்ஸரோவ்யால:}
% {ப்ரத்யய: ஸர்வதர்ச’ன:}

% \SlokaLarge
% {அஜஸ்: ஸர்வேச்’வரஸ்: ஸித்த:}
% {ஸித்திஸ்:‌ ஸர்வாதிரச்யுத:}
% {வ்ருஷாகபிரமேயாத்மா}
% {ஸர்வயோக வினிஸ்ருத:}

% \SlokaLarge
% {வஸுர்‌ வஸுமனாஸ்: ஸத்யஸ்:}
% {ஸமாத்மா ஸம்மிதஸ்ஸம:}
% {அமோக: புண்டரீகாக்ஷோ}
% {வ்ருஷகர்மா வ்ருஷாக்ருதி:}

% \SlokaLarge
% {ருத்ரோ பஹுசிரா பப்ருர்‌}
% {விச்’வயோனி: சு’சிச்ரவா:}
% {அம்ருத: சா’ச்’வதஸ்தாணுர்‌}
% {வராரோஹோ மஹாதபா:}

% \SlokaLarge
% {ஸர்வக: ஸர்வவித்‌ பானுர்‌}
% {விஷ்வக்ஸேனோஜநார்தன:}
% {வேதோ வேதவிதவ்யங்கோ}
% {வேதாங்கோ வேதவித்‌கவி:}

% \SlokaLarge
% {லோகாத்யக்ஷ: ஸுராத்யக்ஷோ}
% {தர்மாத்யக்ஷ: க்ருதாக்ருத:}
% {சதுராத்மா சதுர்வ்யூஹ:}
% {சதுர்‌தம்ஷ்ட்ரச்‌ சதுர்ப்புஜ:}

% \SlokaLarge
% {ப்ராஜிஷ்ணுர்‌ போஜனம்‌ போக்தா}
% {ஸஹிஷ்ணுர்‌ ஜகதாதிஜ:}
% {அனகோ விஜயோ ஜேதா}
% {விச்’வயோனி: புனர்வஸு:}

% \SlokaLarge
% {உபேந்த்ரோ வாமன: ப்ராம்சு’:}
% {அமோக: சு’சிரூர்ஜித:}
% {அதீந்த்ர:ஸங்க்ரஹ: ஸர்கோ}
% {த்ருதாத்மா நியமோயம:}

% \SlokaLarge
% {வேத்யோ வைத்ய: ஸதா யோகீ}
% {வீரஹா மாதவோ மது:}
% {அதீந்த்ரியோ மஹாமாயோ}
% {மஹோத்ஸாஹோ மஹாபல:}

% \SlokaLarge
% {மஹா புத்திர்‌ மஹாவீர்யோ}
% {மஹாச’க்திர்‌ மஹாத்யுதி:}
% {அநிர்த்தேச்’யவபு:}
% {ஸ்ரீமான்அமேயாத்மா மஹாத்ரித்ருக்}

% \SlokaLarge
% {மஹேஷ்வாஸோ மஹீபர்த்தா}
% {ஶ்ரீநிவாஸ:ஸதாங்கதி:}
% {அநிருத்த: ஸுராநந்தோ}
% {கோவிந்தோகோவிதாம்‌ பதி:}

% \SlokaLarge
% {மரீசிர்‌ தமனோஹம்ஸ:}
% {ஸுபர்ணோ புஜகோத்தம:}
% {ஹிரண்யநாப: ஸுதபா:}
% {பத்மநாப: ப்ரஜாபதி:}

% \SlokaLarge
% {அம்ருத்யு: ஸர்வத்ருக்‌ ஸிம்ஹ:}
% {ஸந்தாதா ஸந்திமானம்‌ ஸ்த்திர:}
% {அஜோ துர்மர்ஷண: சா’ஸ்தா}
% {விச்’ருதாத்மா ஸுராரிஹா}

% \SlokaLarge
% {குருர்‌ குருதமோ தாம;}
% {ஸத்ய: ஸத்ய: பராக்ரம:}
% {நிமிஷோ(அ)நிமிஷ: ஸ்ரக்வீ}
% {வாசஸ்பதி ருதாரதீ:}

% \SlokaLarge
% {அக்ரணீர்—க்ராமணீ: ஸ்ரீமான்‌}
% {ந்யாயோ நேதா ஸமீரண:}
% {ஸஹஸ்ரமூர்த்தாவிச்’வாத்மா‌}
% {ஸஹஸ்ராக்ஷ: ஸஹஸ்ரபாத்‌}

% \SlokaLarge
% {ஆவர்த்தனோ நிவ்ருத்தாத்மா}
% {ஸம்வ்ருத: ஸம்ப்ரமர்த்தன:}
% {அஹ:ஸம்வர்த்தகோ வஹ்னி—ரநிலோ தரணீதர:}

% \SlokaLarge
% {ஸுப்ரஸாத: ப்ரஸந்நாத்மா}
% {விச்’வத்ருக்‌ விச்’வபுக்‌ விபு:}
% {ஸத்கர்த்தா ஸத்க்ருதஸ்: ஸாதூர்‌}
% {ஜஹ்னுர்‌ நாராயணோநர:}

% \SlokaLarge
% {அஸங்க்யேயோ (அ)ப்ரமேயாத்மா}
% {விசிஷ்ட: சி’ஷ்டக்ருச்‌சு’சி: /}
% {ஸித்தார்த்த: ஸித்தஸங்கல்ப:}
% {ஸித்தித: ஸித்தி ஸாதன:}

% \SlokaLarge
% {வ்ருஷாஹீ வ்ருஷபோ விஷ்ணுர்‌}
% {வ்ருஷபர்வா வ்ருஷோதர:}
% {வர்த்தனோ வர்த்தமானச்’‌ ச}
% {விவிக்த: ச்’ருதி ஸாகர:}

% \SlokaLarge
% {ஸுபுஜோ துர்த்தரோ வாக்மீ}
% {மஹேந்த்ரோ வஸுதோ வஸு:}
% {நைகரூபோ ப்ருஹத்ரூப:}
% {சி’பிவிஷ்ட: ப்ரகாச’ன: //29}

% {ஓஜஸ்‌தேஜோத்யுதிதர:}
% {ப்ரகாசா’த்மா ப்ரதாபன: /}
% {ருத்த: ஸ்பஷ்டாக்ஷரோ மந்த்ர:}
% {சந்த்ராம்சு’ர்‌ பாஸ்கரத்யுதி:}

% \SlokaLarge
% {அம்ருதாம்சூ’த்பவோ பானு:}
% {ச’ச’பிந்து: ஸூரேச்’வர:}
% {ஒளஷதம்‌ ஜகத: ஸேது:}
% {ஸத்ய தர்ம பராக்ரம:}

% \SlokaLarge
% {பூதபவ்ய பவந்நாத:}
% {பவன: பாவனோ(அ)நல:}
% {காமஹா காமக்ருத்‌ காந்த:}
% {காம: காமப்ரத: ப்ரபு:}

% \SlokaLarge
% {யுகாதிக்ருத்‌ யுகாவர்த்தோ}
% {நைகமாயோ மஹாச’ன:}
% {அத்ருச்’யோவ்யக்தரூபச்’ச}
% {ஸஹஸ்ரஜிதனந்தஜித்}

% \SlokaLarge
% {இஷ்டோஷ்விசி’ஷ்ட: சி’ஷ்டேஷ்ட:}
% {சி’கண்டீ நஹுஷோவ்ருஷ:}
% {க்ரோதஹா க்ரோதக்ருத்‌ கர்த்தா}
% {விச்’வபாஹுர்‌ மஹீதர:}

% \SlokaLarge
% {அச்யுத: ப்ரதித: ப்ராண:}
% {ப்ராணதோ வாஸவாநனுஜ:}
% {அபாம்நிதிரதிஷ்ட்டான}
% {மப்ரமத்த: ப்ரதிஷ்ட்டித:}

% \SlokaLarge
% {ஸ்கந்த: .ஸ்கந்ததரோதுர்யோ}
% {வரதோ வாயுவாஹன:}
% {வாஸுதேவோ ப்ருஹத்பானு}
% {ராதிதேவ: புரந்தர:}

% \SlokaLarge
% {அசோ’கஸ்‌ தாரணஸ்—தார:}
% {சூ’ர‌ செ’ளரிர்‌ ஜனேச்’வர:}
% {அனுகூல:‌ ச’தாவர்த்த:}
% {பத்மீ பத்மநிபேக்ஷண:}

% \SlokaLarge
% {பத்மநாபோ(அ)ரவிந்தாக்ஷ:}
% {பத்மகர்ப்ப: ச’ரீரப்ருத்}
% {மஹர்த்திர்ருத்தோ வ்ருத்தாத்மா}
% {மஹாக்ஷோ கருடத்வஜ:}

% \SlokaLarge
% {அதுல: ச’ரபோ பீம:}
% {ஸமயஜ்ஞோ ஹவிர்‌ஹரி:}
% {ஸர்வலக்ஷண லக்ஷண்யோ}
% {லக்ஷ்மீவான்ஸமிதிஞ்ஜய:}

% \SlokaLarge
% {விக்ஷரோ ரோஹிதோ மார்க்கோ}
% {ஹேதுர்‌ தாமோதர: ஸஹ:}
% {மஹீதரோ மஹாபாகோ}
% {வேகவாநமிதாசன:}

% \SlokaLarge
% {உத்பவ: க்ஷோபணோதேவ:}
% {ஸ்ரீகர்ப்ப: பரமேச்வர:}
% {கரணம்‌ காரணம்‌ கர்த்தா}
% {விகர்த்தா கஹனோ குஹ:}

% \SlokaLarge
% {வ்யவஸாயோவ்யவஸ்த்தான:}
% {ஸம்ஸ்த்தான: ஸ்த்தானதோத்ருவ:}
% {பரர்த்தி: பரமஸ்பஷ்ட :‌}
% {துஷ்ட: புஷ்ட: சு’பேக்ஷண:}

% \SlokaLarge
% {ராமோ விராமோ விரதோ}
% {மார்கோ நேயோ நயோ(அ)நய:}
% {வீர: ச’க்திமதாம்‌ ச்’ரேஷ்ட்டோ}
% {தர்மோ தர்மவிதுத்தம:}

% \SlokaLarge
% {வைகுண்ட்ட: புருஷ: ப்ராண:}
% {ப்ராணத: ப்ரணவ: ப்ருது:}
% {ஹிரண்யகர்ப்ப: ச’த்ருக்னோ}
% {வ்யாப்தோ வாயுரதோக்ஷஜ:}

% \SlokaLarge
% {ருது : ஸுதர்சன: கால:}
% {பரமேஷ்ட்டீபரிக்ரஹ:}
% {உக்ர: ஸம்வத்ஸரோ தக்ஷோ}
% {விச்’ராமோ விச்’வதக்ஷிண:}

% \SlokaLarge
% {விஸ்தார: ஸ்த்தாவரஸ்தாணு:}
% {ப்ரமாணம்‌ பீஜ மவ்யயம்‌}
% {அர்த்தோ(அ)னர்த்தோ மஹாகோசோ}
% {மஹாபோகோ மஹாதன:}

% \SlokaLarge
% {அநிர்விண்ண: ஸ்த்தவிஷ்டோ(அ)பூர்‌—}
% {தர்மயூபோ மஹாமக:}
% {நக்ஷத்ரநேமிர்‌—நக்ஷத்ரீ}
% {க்ஷம: க்ஷாம: ஸமீஹன:}

% \SlokaLarge
% {யஜ்ஞ இஜ்யோ மஹேஜ்யச்’ச}
% {க்ரது: ஸத்ரம்‌ ஸதாங்கதி:}
% {ஸர்வதர்சீ’ விமுக்தாத்மா}
% {ஸர்வஜ்ஞோ ஜ்ஞானமுத்தமம்}

% \SlokaLarge
% {ஸுவ்ரத: ஸுமுக: ஸூக்ஷ்ம:}
% {ஸுகோஷ: ஸுகத: ஸுஹ்ருத்}
% {மநோஹரோ ஜிதக்ரோதோ}
% {வீரபாஹுர்‌ விதாரண:}

% \SlokaLarge
% {ஸ்வாபன: ஸ்வவசோ’ வ்யாபீ}
% {நைகாத்மா நைககர்மக்ருத்‌}
% {வத்ஸரோ வத்ஸலோ வத்ஸீ}
% {ரத்னகர்ப்போ தனேச்’வர:}

% \SlokaLarge
% {தர்மகுப்‌ தர்மக்ருத்‌ தர்மீ}
% {ஸ—தஸத்க்ஷரமக்ஷரம்‌ /}
% {அவிஜ்ஞாதா ஸஹஸ்ராம்சு’ர்—}
% {விதாதா க்ருதலஷண:}

% \SlokaLarge
% {கபஸ்திநேமி: ஸத்வஸ்த்த:}
% {ஸிம்ஹோ பூதமஹேச்’வர:}
% {ஆதிதேவோ மஹாதேவோ}
% {தேவேசோ’ தேவப்ருத்‌ குரு:}

% \SlokaLarge
% {உத்தரோ கோபதிர்‌ கோப்தா}
% {க்ஞானகம்ய: புராதன:}
% {ச’ரீரபூதப்ருத்‌ போக்தா}
% {கபீந்த்ரோ பூரிதஷிண:}

% \SlokaLarge
% {ஸோமபோ(அ)ம்ருதப: ஸோம:}
% {புருஜித்‌ புருஸத்தம:}
% {விநயோ ஜய: ஸத்யஸந்தோ}
% {தாசா’ர்ஹ: ஸாத்வதாம்‌ பதி:}

% \SlokaLarge
% {ஜீவோ விநயிதா ஸாக்ஷீ}
% {முகுந்தோ(அ)மிதவிக்ரம: /}
% {அம்போநிதிரனந்தாத்மா}
% {மஹோததிச’யோ(அ)ந்தக:}

% \SlokaLarge
% {அஜோ மஹார்ஹ: ஸ்வாபாவ்யோ}
% {ஜிதாமித்ர: ப்ரமோதன: /}
% {ஆனந்தோ நந்தனோ நந்த:}
% {ஸத்யதர்மா த்ரிவிக்ரம:}

% \SlokaLarge
% {மஹா்ஷி: கபிலாசார்ய:}
% {க்ருதஜ்ஞோ மேதினீபதி:}
% {த்ரிபதஸ்த்ரிதசா’த்யக்ஷோ}
% {மஹாச்’ருங்க: க்ருதாந்தக்ருத்}

% \SlokaLarge
% {மஹாவராஹோ கோவிந்த:}
% {ஸுஷேண: கனகாங்கதீ}
% {குஹ்யோகபீரோ கஹனோ}
% {குப்தச்’‌ சக்ர கதாதர:}

% \SlokaLarge
% {வேதா: ஸ்வாங்கோ(அ)ஜித: க்ருஷ்ணோ}
% {த்ருட: ஸங்கர்ஷணோ(அ)ச்’யுத:}
% {வருணோ வாருணோ வ்ருக்ஷ:}
% {புஷ்கராக்ஷோ மஹாமனா:}

% \SlokaLarge
% {பகவான்‌ பகஹா(அ)நந்தீ}
% {வநமாலீ ஹலாயுத:}
% {ஆதித்யோ ஜ்யோதிராதித்ய:}
% {ஸஹிஷ்ணுர்கதிஸத்தம:}

% \SlokaLarge
% {ஸுதன்வா கண்டபரசுர்‌}
% {தாருணோ த்ரவிணப்ரத: /}
% {திவஸ்ப்ருக்‌ ஸர்வத்ருக்‌வ்யாஸோ}
% {வாசஸ்பதிரயோநிஜ:}

% \SlokaLarge
% {த்ரிஸாமா ஸாமக: ஸாம}
% {நிர்வாணம்‌ பேஷஜம்‌ பிஷக்}
% {ஸந்யாஸக்ருச்‌சம: சா’ந்தோ}
% {நிஷ்ட்டா சா’ந்தி: பராயணம்‌}

% \SlokaLarge
% {சு’பாங்க: சா’ந்தித: ஸ்ரஷ்டா}
% {குமுத: குவலேச’ய:}
% {கோஹிதோகோபதிர்‌ கோப்தா}
% {வ்ருஷபாக்ஷோ வ்ருஷப்ரிய:}

% \SlokaLarge
% {அநிவர்த்தீ நிவ்ருத்தாத்மா}
% {ஸம்க்ஷேப்தா க்ஷேமக்ருச்சிவ:}
% {ஸ்ரீவத்ஸவஷா: ஸ்ரீவாஸ:}
% {ஸ்ரீபதி: ஸ்ரீமதாம்‌ வர:}

% \SlokaLarge
% {ஸ்ரீத: ஸ்ரீச’: ஸ்ரீநிவாஸ:}
% {ஸ்ரீநிதி: ஸ்ரீவிபாவன:}
% {ஸ்ரீதர: ஸ்ரீகர: ச்’ரேய:}
% {ஸ்ரீமான் லோகத்ரயாச்’ரய:}

% \SlokaLarge
% {ஸ்வக்ஷ: ஸ்வங்க: ச’தானந்தோ}
% {நந்திர்‌ஜ்யோதிர்கணேச்’வர:}
% {விஜிதாத்மா(அ)விதேயாத்மா}
% {ஸத்கீர்த்திச்’‌ சின்னஸம்ச’ய :}

% {உதீர்ண: ஸர்வதச்’சக்ஷு}
% {ரனீச’: சா’ச்வதஸ்த்திர:}
% {பூச’யோ பூஷணோ பூதிர்‌}
% {விசோ’க: சோகநாச’ன:}

% \SlokaLarge
% {அர்ச்சிஷ்மானர்ச்சித: கும்போ}
% {விசு’த்தாத்மா விசோ’தன:}
% {அநிருத்தோ(அ)ப்ரதிரத:}
% {ப்ரத்யும்னோ(அ)மிதவிக்ரம :}

% \SlokaLarge
% {காலநேமிநிஹா வீர:}
% {செள’ரி: சூ’ர ஜனேச்’வர:}
% {த்ரிலோகாத்மா த்ரிலோகேச’:}
% {கேச’வ: கேசி’ஹா ஹரி:}

% \SlokaLarge
% {காமதேவ: காமபால:}
% {காமீ காந்த: க்ருதாகம:}
% {அநிர்தேச்’யவபுர்‌ விஷ்ணுர்‌}
% {வீரோ(அ)னந்தோ தனஞ்ஜய:}

% \SlokaLarge
% {ப்ரஹ்மண்யோ ப்ரஹ்மக்ருத்‌ ப்ரஹ்மா}
% {ப்ரஹ்ம ப்ரஹ்மவிவர்த்தந:}
% {ப்ரஹ்மவித்‌ ப்ராஹ்மணோ ப்ரஹ்மீ}
% {ப்ரஹ்மஜ்ஞோ ப்ராஹ்மணப்ரிய:}

% \SlokaLarge
% {மஹாக்ரமோ மஹாகர்மா}
% {மஹாதேஜா மஹோரக:}
% {மஹாக்ரதுர்‌ மஹாயஜ்வா}
% {மஹாயஜ்ஞோ மஹாஹவி:}

% \SlokaLarge
% {ஸ்தவ்ய: ஸ்தவப்ரிய: ஸ்தோத்ரம்‌}
% {ஸ்துதி: ஸ்தோதாரணப்ரிய:}
% {பூர்ண: பூரயிதா புண்ய:}
% {புண்யகீர்த்திரநாமய:}

% \SlokaLarge
% {மனோஜவஸ்‌ தீர்த்தகரோ}
% {வஸுரேதா வஸுப்ரத:}
% {வஸுப்ரதோ வாஸுதேவோ}
% {வஸுர்‌ வஸுமனா ஹவி:}

% \SlokaLarge
% {ஸத்கதி: ஸத்க்ருதி: ஸத்தா}
% {ஸத்பூதி: ஸத்பராயண:}
% {சூ’ரஸேனோ யதுச்’ரேஷ்ட:}
% {ஸந்நிவாஸ: ஸுயாமுன:}

% \SlokaLarge
% {பூதாவாஸோ வாஸுதேவ:}
% {ஸர்வாஸு நிலயோ(அ)னல:}
% {தர்ப்பஹா தர்ப்பதோத்ருப்தோ}
% {துர்த்தரோ(அ)தா(அ)பராஜித:}

% \SlokaLarge
% {விச்’வ மூர்த்திர்‌—மஹா மூர்த்திர்‌—}
% {தீப்தமூர்த்தி—ரமூர்த்திமான்‌ /}
% {அநேகமூர்த்தி—ரவ்யக்த:}
% {ச’தமூர்த்தி: சதானன:}

% \SlokaLarge
% {ஏகோ நைக: ஸவ: க: கிம்‌}
% {யத்தத்‌ பதமனுத்தமம்}
% {லோகபந்துர்‌ லோகநாதோ}
% {மாதவோபக்தவத்ஸல:}

% \SlokaLarge
% {ஸுவர்ணவர்ணோ ஹேமாங்கோ}
% {வராங்கச்’‌ சந்தனாங்கதீ /}
% {வீரஹா விஷம: சூ’ன்யோ}
% {க்ருதாசீ’ரசலச்’‌ சல:}

% \SlokaLarge
% {அமானீமானதோ மான்யோ}
% {லோகஸ்வாமீ த்ரிலோகத்ருக்}
% {ஸுமேதா மேதஜோ தன்ய:}
% {ஸத்யமேதா தராதர:}

% \SlokaLarge
% {தேஜோவ்ருஷோ த்யுதிதர:}
% {ஸர்வச’ஸ்த்ரப்ருதாம்‌ வர:}
% {ப்ரக்ரஹோ நிக்ரஹோவ்யக்ரோ}
% {நைகச்’ருங்கோ கதாக்ரஜ:}

% \SlokaLarge
% {சதுர்‌மூர்த்திச்‌ சதுர்ப்பாஹுச்‌}
% {சதுர்வ்யூஹஸ்சதுர்கதி:}
% {சதுராத்மா சதுர்ப்பாவச்‌}
% {சதுர்வேத விதேகபாத்‌}

% \SlokaLarge
% {ஸமாவர்த்தோ(அ)நிவ்ருத்தாத்மா}
% {துர்ஜயோ துரதி க்ரம:}
% {துர்லபோ துர்கமோ துர்க்கோ}
% {துராவாஸோ துராரிஹா}

% \SlokaLarge
% {சு’பாங்கோ லோகஸாரங்க:}
% {ஸுதந்துஸ்தந்துவர்த்தன:}
% {இந்த்ரகர்மா மஹாகர்மா}
% {க்ருதகர்மா க்ருதாகம:}

% \SlokaLarge
% {உத்பவ: ஸுந்தர: ஸுந்தோ}
% {ரத்நநாப: ஸுலோசன:}
% {அர்க்கோ வாஜஸனச்’ருங்கீ}
% {ஜயந்த்த: ஸர்வவிஜ்ஜயீ}

% \SlokaLarge
% {ஸுவர்ணபிந்து ரக்ஷோப்ய:}
% {ஸர்வ வாகீச்’வரேச்’ வர:}
% {மஹாஹ்ரதோ மஹாகர்த்தோ}
% {மஹாபூதோ மஹாநிதி:}

% \SlokaLarge
% {குமுத: குந்தர: குந்த:}
% {பர்ஜன்ய: பாவனோ(அ)நில:}
% {அம்ருதாம்சோ(அ)ம்ருதவபு:}
% {ஸர்வஜ்ஞ: ஸர்வதோமுக:}

% \SlokaLarge
% {ஸுலப: ஸுவ்ரத: ஸித்த:}
% {ச’த்ருஜிச்‌—ச’த்ருதாபன: /}
% {நயக்ரோதோதும்பரோ(அ)ச்வத்த}
% {ச்சாணூராந்த்ர நிஷூதன:}

% \SlokaLarge
% {ஸஹஸ்ரார்ச்சி: ஸப்தஜிஹ்வ:}
% {ஸப்தைதா: ஸப்தவாஹன:}
% {அமூர்த்திரனகோ(அ)சிந்த்யோ}
% {பயக்‌ருத்‌ பயநாசன:}

% \SlokaLarge
% {அணுர்‌ ப்ருஹத்‌ க்ருச’: ஸ்த்தூலோ}
% {குணப்ருந்‌நிர்குணோமஹான்‌}
% {அத்ருத: ஸ்வத்ருத; ஸ்வாஸ்ய:}
% {ப்ராக்வம்சோ வம்சவர்த்தன:}

% \SlokaLarge
% {பாரப்ருத்‌ கதிதோ யோகீ}
% {யோகீச’: ஸர்வகாமத:}
% {ஆச்’ரம: ச’ரமண: க்ஷாம:}
% {ஸுபர்ணோ வாயுவாஹன:}

% \SlokaLarge
% {தனுர்த்தரோ தனுர்வேதோ}
% {தண்டோ தமயிதாதம:}
% {அபராஜித: ஸர்வஸஹோ}
% {நியந்தா(அ)நியமோ(அ)யம:}

% \SlokaLarge
% {ஸத்வவான் ஸாத்விக: ஸத்ய:}
% {ஸத்யதர்ம பராயண:}
% {அபிப்ராய: ப்ரியார்ஹோ(அ)ர்ஹ:}
% {ப்ரியக்ருத்‌ ப்ரீதி வர்த்தன:}

% \SlokaLarge
% {விஹாயஸகதிர்‌—ஜ்யோதி:}
% {ஸூருசிர்‌—ஹுதபுக்‌ விபு:}
% {ரவிர்விரோச’ன: ஸூர்ய:}
% {ஸவிதா ரவிலோசன:}

% \SlokaLarge
% {அனந்தோ ஹுதபுக்‌போக்தா}
% {ஸுகதோ நைகஜோ(அ)க்ரஜ:}
% {அதிர்விண்ண: ஸதாமர்ஷீ}
% {லோகாதிஷ்ட்டானமத்புத:}

% \SlokaLarge
% {ஸநாத்‌ ஸநாதனதம:}
% {கபில: கபிரவ்யய:}
% {ஸ்வஸ்தித: ஸ்வஸ்திக்ருத்‌ ஸ்வஸ்தி}
% {ஸ்வஸ்திபுக்‌ ஸ்வஸ்தி தக்ஷிண:}

% \SlokaLarge
% {அரெளத்ர: குண்டலீ சக்ரீ}
% {விக்ரம்யூர்ஜிதசாஸன:}
% {ச’ப்தாதிக: ச’ப்தஸஹ:}
% {சி’சிர: ச’ர்வரீகர:}

% \SlokaLarge
% {அக்ரூர: பேசலோ தக்ஷோ}
% {தக்ஷிண: க்ஷமிணாம்வர:}
% {வித்வத்தமோ வீதபய:}
% {புண்யச்’ரவண கீர்த்தன:}

% \SlokaLarge
% {உத்தாரணோ துஷ்க்ருதிஹா}
% {புண்யோ து: ஸ்வப்னநாசன:}
% {வீரஹா ரக்ஷண: ஸந்தோ}
% {ஜீவன: பர்யவஸ்த்தித:}

% \SlokaLarge
% {அனந்தரூபோ(அ)னந்தஸ்ரீர்‌}
% {ஜித மன்யுர்‌ பயாபஹ:}
% {சதுரச்’ரோ கபீராத்மா}
% {விதிசோ’ வ்யாதிசோ’ திச’:}

% \SlokaLarge
% {அனாதிர்‌ பூர்ப்புவோ லக்ஷ்மீ:}
% {ஸுவீரோ ருசிராங்கத:}
% {ஜனனோ ஜன்ஜன்மாதிர்‌}
% {பீமோ பீமபராக்ரம:}

% \SlokaLarge
% {ஆதாரநிலயோ(அ)தாதா}
% {புஷ்பஹாஸ: ப்ரஜாகர:}
% {ஊர்த்வக: ஸத்பதாசார:}
% {ப்ராணத: ப்ரணவ: பண:}

% \SlokaLarge
% {ப்ரமாணம்‌ ப்ராணநிலய:}
% {ப்ராணப்ருத்‌ ப்ராணஜீவன:}
% {தத்வம்‌ தத்வவிதேகாத்மா}
% {ஜன்மம்ருத்யு ஐராதிக:}

% \SlokaLarge
% {பூர்ப்புவ: ஸ்வஸ்தருஸ்‌தார:}
% {ஸவிதா ப்ரபிதாமஹ:}
% {யஜ்ஞோ யஜ்ஞபதிர்‌ யஜ்வா}
% {யஜ்ஞாங்கோ யஜ்ஞவாஹன:}

% \SlokaLarge
% {யஜ்ஞப்ருத்‌யஜ்ஞக்ருத்‌ யஜ்ஞீ}
% {யஜ்ஞபுக்‌—யஜ்ஞஸாதன:}
% {யஜ்ஞாந்தக்ருத்‌—யஜ்ஞகுஹ்ய—}
% {மன்ன—மன்னாத ஏவ ச}

% \SlokaLarge
% {ஆத்மயோனி: ஸ்வயம்ஜாதோ}
% {வைகாந: ஸாமகாயன:}
% {தேவகீ நந்தன: ஸ்ரஷ்டா}
% {க்ஷிதீச’: பாபநாச’ன:}

% \SlokaLarge
% {ச’ங்கப்ருந்நந்தகீ சக்ரீ}
% {சா’ர்ங்கதன்வா கதாதர:}
% {ரதாங்கபாணி ரக்ஷோப்ய:}
% {ஸர்வ ப்ரஹரணாயுத:}

% \SlokaLarge
% {ஸர்வ ப்ரஹரணாயுத ஒம்‌ நம இதி}
% {வனமாலீ கதீ சா’ர்ங்கீ}
% {ச’ங்கீ சக்ரீ ச நந்தகீ}
% {ஸ்ரீமான்நாராயணோ விஷ்ணுர்‌}
% {வாஸுதேவோ(அ)பிரக்ஷது}
% {(என்று 3 தடவை சொல்லவும்‌)}

% \SlokaLarge
% {பலச்ருதி}
% {இதீதம்‌ கீர்த்தனீயஸ்ய}
% {கேச’வஸ்ய மஹாத்மன:}
% {நாம்னாம் ‌ஸஹஸ்ரம்‌ திவ்யானாம்‌}
% {அசே’ஷேண ப்ரகீர்த்திதம் ‌}

% \SlokaLarge
% {ய இதம்‌ ச்’ருணுயாந்‌நித்யம்‌}
% {யச்’சாபி பரிகீர்த்தயேத்‌}
% {நாசு’பம் ப்ராப்னுயாத் ‌கிஞ்சித்‌}
% {ஸோ(அ) முத்ரேஹ ச மானவ:}

% \SlokaLarge
% {வேதாந்தகோ ப்ராஹ்மண: ஸ்யாத்‌}
% {க்ஷத்ரியோ விஜயீ பவேத்‌}
% {வைச்’யோ தன—ஸம்ருத்த: ஸ்யாத்}
% {சூ’த்ர: ஸுக மவாப்னுயாத்‌}

% \SlokaLarge
% {தர்மார்த்தீ ப்ராப்னுயாத்‌தர்ம}
% {மர்த்தார்த்தீ சார்த்த மாப்னுயாத்‌}
% {காமான—வாப்னுயாத் ‌காமீ}
% {ப்ரஜார்த்தீ சாப்னுயாத்‌ ப்ரஜாம்‌}

% \SlokaLarge
% {பக்திமான் ய: ஸதோத்தாய}
% {சு’சி ஸ்‌தத்கதமானஸ:}
% {ஸஹஸ்ரம்‌ வாஸுதேவஸ்ய}
% {நாம்னா—மேதத்‌ ப்ரகீர்த்தயேத் ‌}

% \SlokaLarge
% {யச’: ப்ராப்னோதி விபுலம்‌}
% {யாதி ப்ராதான்யமேவ ச}
% {அசலாம்‌ ச்’ரியமாப்னோதி}
% {ச்’ரேய: ப்ராப்னோத்ய னுத்தமம்}

% \SlokaLarge
% {ந பயம்‌ க்வசிதாப்னோதி}
% {வீர்யம் ‌தேஜச்’ ச விந்ததி /}
% {பவத்யரோகோ த்யுதிமான்}
% {பலரூப குணான்வித:}

% \SlokaLarge
% {ரோகார்தோ முச்யதே ரோகாத்‌}
% {பத்தோ முச்யேத பந்தனாத்‌}
% {பயான் முச்யேத பீதஸ்து}
% {முச்யேதாபன்ன ஆபத:}

% \SlokaLarge
% {துர்காண்யதிதர த்‌யாசு}
% {புருஷ: புருஷோத்தமம்‌}
% {ஸ்துவந்‌நாம ஸஹஸ்ரேண}
% {நித்யம் ‌பக்தி ஸமன்வித:}

% \SlokaLarge
% {வாஸுதேவாச்’ரயோ மர்த்யோ}
% {வாஸுதேவ பராயண:}
% {ஸர்வபா பவிசு’த்தாத்மா}
% {யாதி ப்ரஹ்ம ஸநாதனம்‌}

% \SlokaLarge
% {ந வாஸு தேவ பக்தானாம்‌}
% {அசு’பம்‌ வித்யதே க்வசித்‌}
% {ஜன்ம ம்ருத்யு ஜராவ்யாதி}
% {பயம்‌ நைவோ பஜாயதே}

% \SlokaLarge
% {இமம்‌ ஸ்தவமதீயான:}
% {ச்’ரத்தாபக்தி ஸமன்வித:}
% {யுஜ்யேதாத்ம ஸுகக்ஷாந்தி}
% {ஶ்ரீத்ருதி: ஸ்ம்ருதி கீர்த்திபி:}

% \SlokaLarge
% {ந க்ரோதோ ந ச மாத்ஸர்யம்‌}
% {ந லோபோ நாசு’பாமதி:}
% {பவந்தி க்ருதபுண்யானாம்‌}
% {பக்தானாம்‌ புருஷோத்தமே}

% \SlokaLarge
% {த்யெள: ஸ சந்த்ரார்க்க நக்ஷத்ரா}
% {கம் திசோ’ பூர்‌மஹோததி:}
% {வாஸுதேவஸ்ய வீர்யேண}
% {வித்ருதானி மஹாத்மன:}

% \SlokaLarge
% {ஸஸுராஸுர கந்தர்வம்‌}
% {ஸயக்ஷோரக ராக்ஷஸம்‌}
% {ஜகத்வசே’ வர்த்ததேதம்‌}
% {க்ருஷ்ணஸ்ய ஸ சராசரம் ‌}

% \SlokaLarge
% {இந்த்ரியாணி மனோபுத்தி:}
% {ஸத்வம்‌ தேஜோ பலம்‌ த்ருதி:}
% {வாஸுதேவாத்ம கான்யாஹூ:}
% {க்ஷேத்ரம்‌ க்ஷேத்ரஜ்ஞ ஏவ ச}

% \SlokaLarge
% {ஸர்வாகமானா மாசார:}
% {ப்ரதமம் ‌பரிகல்பதே}
% {ஆசார ப்ரபவோ தர்மோ}
% {தர்மஸ்ய ப்ரபுரச்யுத:}

% \SlokaLarge
% {ருஷய: பிதரோ தேவா:}
% {மஹாபூதானி தாதவ:}
% {ஜங்கமா ஜங்கமம்‌ சேதம்‌}
% {ஜகந்‌நாராயணோத்பவம்}

% \SlokaLarge
% {யோகோஜ்ஞானம்‌ ததா ஸாங்க்யம்‌}
% {வித்யா: சி’ல்பாதிகர்மச}
% {வேதா: சா’ஸ்த்ராணி விஜ்ஞானம்‌}
% {ஏதத் ‌ஸர்வம் ‌ஜனார்த்தனாத் ‌}

% \SlokaLarge
% {ஏகோ விஷ்ணுர்‌ மஹத் பூதம்‌}
% {ப்ருதக்‌பூதா ன்யநேகச’:}
% {த்ரீன்லோகான் வ்யாப்ய பூதாத்மா}
% {புங்க்தே விச்’வபுகவ்யய:}

% \SlokaLarge
% {இமம் ‌ஸ்தவம்‌ பகவதோ}
% {விஷ்ணோர் ‌வ்யாஸேன கீர்த்திதம்‌ /}
% {படேத்ய இச்சேத் ‌புருஷ:}
% {ச்’ரேய: ப்ராப்தும் ‌ஸுகானி ச}

% \SlokaLarge
% {விச்’வேச்’வரமஜம்‌ தேவம்‌}
% {ஜகத: ப்ரபுமவ்யயம்‌}
% {பஜந்தி யே புஷ்கராக்ஷம்‌}
% {ந தே யாந்தி பராபவம்‌}

% \SlokaLarge
% {ந தே யாந்தி பராபவ ஓம் நம இதி}
% {அர்ஜுன உவாச}
% {பத்மபத்ர விசா’லாக்ஷ}
% {பத்மநாப ஸுரோத்தம}
% {பக்தானா மனுரக்தானாம்‌}
% {த்ராதா பவ ஜநார்த்தன}

% \SlokaLarge
% {ஸ்ரீ பகவானுவாச—}
% {யோ மாம்‌ நாம ஸஹஸ்ரேண}
% {ஸ்தோதுமிச்சதி பாண்டவ}
% {ஸோ(அ)ஹமேகேன ச்’லோகேன}
% {ஸ்துத ஏவ ந ஸம்ச’ய:}

% \SlokaLarge
% {ஸ்துத ஏவ ந ஸம்ச’ய ஓம்‌ நம இதி}
% {வ்யாஸ உவாச—}
% {வாஸனாத்‌ வாஸுதேவஸ்ய}
% {வாஸிதம்‌ புவனத்ரயம்‌}
% {ஸர்வபூத நிவாஸோ(அ)ஸி}
% {வாஸுதேவ நமோ(அ)ஸ்துதே}

% \SlokaLarge
% {ஸ்ரீ வாஸுதேவ நமோஸ்துத ஒம்‌ நம இதி}
% {ஸ்ரீ பார்வத்யுவாச—}
% {கேனோபாயேன லகுனா}
% {விஷ்ணோர்‌ நாம ஸஹஸ்ரகம்‌ /}
% {பட்யதே பண்டிதைர்‌ நித்யம்}
% {ச்’ரோதுமிச்சாம்யஹம்‌ ப்ரபோ}

% \SlokaLarge
% {ஸ்ரீ ஈ’ச்வர உவாச—}
% {ஸ்ரீ ராம ராம ராமேதி}
% {ரமே ராமே மனோரமே}
% {ஸஹஸ்ரநாம தத்துல்யம்‌}
% {ராமநாம வரானனே}

% \SlokaLarge
% {(என்று 3 தடவை சொல்லவும்)}

% \SlokaLarge
% {ஸ்ரீராமநாம வரானன ஓம்‌ நம இதி}
% {ஸ்ரீ ப்ரஹ்மோவாச—}
% {நமோ(அ)ஸ்த்வநனந்தாய ஸஹஸ்ரமூர்த்தயே}
% {ஸஹஸ்ர பாதாக்ஷி சிரோரு பாஹவே}
% {ஸஹஸ்ர நாம்னே புருஷாய சா’ச்வதே}
% {ஸஹஸ்ர கோடி யுகதாரிணே நம:}

% \SlokaLarge
% {ஸ்ரீஸஹஸ்ரகோடி யுகதாரிண ஒம்‌ நம இதி}
% {ஸஞ்ஜய உவாச—}
% {யத்ர யோகேச்’வர: க்ருஷ்ணோ}
% {யத்ர பார்த்தோ தனுர்த்தர:}
% {தத்ர ஸ்ரீர்‌ விஜயோ பூதிர்‌}
% {த்ருவா நீதிர்‌ மதிர் மம}

% \SlokaLarge
% {ஸ்ரீ பகவானுவாச—}
% {அனன்யாஸ் சிந்தயந்தோமாம்}
% {யே ஜனா: பர்யுபாஸதே}
% {தேஷாம் நித்யாபி யுக்தாநாம்}
% {யோகக்ஷேமம் வஹாம்யஹம்}

% \SlokaLarge
% {பரித்ராணாய ஸாதூனாம்‌}
% {விநாசா’ய ச துஷ்க்ருதாம்‌}
% {தர்ம ஸம்ஸ்தாபனார்த்தாய}
% {ஸம்பவாமி யுகே யுகே}

% \SlokaLarge
% {ஆர்த்தா விஷண்ணா: சி’திலாஸ்ச பீதா:}
% {கோரேஷுச வ்யாதிஷு வர்த்தமானா:}
% {ஸங்கீர்த்ய நாராயண ச’ப்த மாத்ரம்‌}
% {விமுக்தது: கா: ஸுகினோ பவந்து}

% \SlokaLarge
% {காயேன வாசா மனஸாஇந்த்ரியைர்‌வா}
% {புத்த்யாத்மனாவா ப்ரக்ருதே: ஸ்வபாவாத்‌}
% {கரோமி யத்யத்‌ ஸகலம்‌ பரஸ்மை}
% {நாராயணாயேதி ஸமர்ப்பயாமி..}

% \phoneticsguide
} \end{document}
