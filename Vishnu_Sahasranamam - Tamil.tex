\documentclass[10pt]{article}
\usepackage[utf8]{inputenc}

% Mult-lingual utility package
\usepackage{polyglossia}
\setdefaultlanguage{english}
\setotherlanguage{tamil}
\defaultfontfeatures{Scale=MatchLowercase}
\newfontfamily\tamilfont{Latha}[Script=Tamil]

% Custom utility package to deal with Sanskrit accents in Tamil
\usepackage{stackmacros}

\begin{document} \texttamil{

\heading{\B{\Huge விஷ்ணு ஸஹஸ்ரநாமம் \\ \eng{\huge (Vishnu Sahasranamam in Tamil)}}}
\newcounter{mycounter}
% śuklāṁbaradharaṁ viṣṇuṁ śaśivarṇaṁ caturbhujam |
% prasannavadanaṁ dhyāyet sarvavighnōpaśāṁtaye || 1 ||
\slokasl
{ஷுக்லா அம்\y{ப}{3}ர\y{த}{4}ர\BU{ம்} விஷ்ணும்}
{ஶஶி \BU{வர்}ணம் சதுர் \y{பு}{4}ஜம்}
{ப்ரஸன்ன வ\y{த}{3}ன\BU{ம்} \y{த்}{4}யாயேத்}
{\BU{ஸ}ர்வ \BU{வி}\y{க்}{4}னோப ஷாந்தயே}

\phantom{\countr}

\BU{\huge பூர்வ பீடிகா \eng{\huge (Poorva Peetika)}}\\

% vyāsaṁ vasiṣṭhanaptāraṁ śakteḥ pautramakalmaṣam |
% parāśarātmajaṁ vaṁde śukatātaṁ tapōnidhim || 3 ||
\slokasl
{வ்யாஸம் வஸிஷ்ட \BU{ந}ப்தாரம்}
{ஶக்தெ\B{:} பௌத்ரம் - அகல்மஷம்}
{பராஷர - ஆத்மஜ\BU{ம்} வந்\y{தே}{3}}
{ஷுக \BU{தா}தம் தபோ நி\y{தி}{4}ம்}

% vyāsāya viṣṇurūpāya vyāsarūpāya viṣṇave |
% namō vai brahmanidhaye vāsiṣṭhāya namō namaḥ || 4 ||
\slokasl
{வ்யாஸாய விஷ்ணு \BU{ரூ}பாய}
{\BU{வ்யா}ஸ \BU{ரூ}பாய விஷ்ணவே}
{நமோ வை ப்ரஹ்ம \BU{நி}\y{த}{4}யே}
{\BU{வா}ஸிஷ்\BU{\y{டா}{2}}ய நமோ நம\B{:}}

% avikārāya śuddhāya nityāya paramātmane |
% sadaikarūparūpāya viṣṇave sarvajiṣṇave || 5 ||
\slokasl
{அவிகாராய \BU{ஷு\y{த்}{3}}\y{தா}{4}ய}
{\BU{நி}த்யாய பரமாத்மனே}
{ஸ\y{தை}{3}க ரூப \BU{ரூ}பாய}
{\BU{வி}ஷ்ணவே ஸர்வ ஜிஷ்ணவே}

% yasya smaraṇamātreṇa janmasaṁsārabaṁdhanāt |
% vimucyate namastasmai viṣṇave prabhaviṣṇave || 6 ||
\slokasl
{யஸ்ய ஸ்மரண \BU{மா}த்ரேண}
{\BU{ஜ}ன்ம \BU{ஸ}ம்ஸார \y{ப}{3}ந்\y{த}{4}நாத்}
{விமுச்யதே ந\BU{ம:} - தஸ்மை \space (ந\BU{மஸ்} / தஸ்மை) }
{\BU{வி}ஷ்ணவே ப்ர\y{ப}{4} விஷ்ணவே}

% ōṁ namō viṣṇave prabhaviṣṇave ||
\slokal
{\B{ஓம்} நமோ \space விஷ்ணவே \space ப்ர\y{ப}{4} \space விஷ்ணவே}

% śrī vaiśaṁpāyana uvāca
% śrutvā dharmānaśeṣeṇa pāvanāni ca sarvaśaḥ |
% yudhiṣṭhiraḥ śāṁtanavaṁ punarevābhyabhyāṣata || 7 ||
\slokadl
{\B{ஶ்ரீ} வைஷம் பாயந உவாச}
{ஶ்ருத்வா \y{த}{4}ர்மாண் - அ\BU{ஷே}ஶேண} {\BU{பா}வ\BU{னா}னி ச ஸர்வஷ:}
{யு\y{தி}{4}ஶ்டிர: ஷாந்தனவம்} {புனரேவ - அ\y{பி}{4} - அ\y{பா}{4}ஶத (அ\y{ப்}{4}ய / \y{பா}{4}ஶத)}

\newpage
% śrī yudhiṣṭhira uvāca
% kimekaṁ daivataṁ lōke kiṁ vāpyekaṁ parāyaṇaṁ |
% stuvaṁtaḥ kaṁ kamarcaṁtaḥ prāpnuyurmānavāḥ śubham || 8 ||
\slokadl
{யு\y{தி}{4}ஶ்டிர உவாச}
{கிம் - ஏகம் \y{தை}{3}வ\BU{தம்} லோகே} {\BU{கி}ம் வாபி - \BU{ஏ}கம் (வாப்\BU{யே} / கம்) பராயணம்}
{ஸ்துவந்த: கம் கம் - அர்சந்த:} {\BU{ப்ரா}ப்னுயு: - மானவ: (ப்ராப்னுயுர்/மானவா) ஷுபம்}

% kō dharmaḥ sarvadharmāṇāṁ bhavataḥ paramō mataḥ |
% kiṁ japanmucyate jaṁturjanmasaṁsārabaṁdhanāt || 9 ||
\slokasl
{கோ \y{த}{4}ர்ம: ஸர்வ \Y{த}{4}ர்மாணாம்}{ \Y{ப}{4}வத: பரமோ மத:}
{கிம் ஜபன் - மமுச்ய\BU{தே} ஜந்து: (ஜந்துர்)}{ \BU{ஜ}ன்ம \BU{ஸ}ம்ஸார \y{ப}{3}ந்\y{த}{4}னாத்}

% śrī bhīṣma uvāca
% jagatprabhuṁ devadevamanaṁtaṁ puruṣōttamam |
% stuvannāmasahasreṇa puruṣaḥ satatōtthitaḥ || 10 ||
\slokadl 
{ஶ்ரீ \y{பீ}{4}ஷ்ம உவாச}
{ஜகத் ப்ர\y{பு}{4}ம் \y{தே}{3}வ \y{தே}{3}வம்}{அநந்தம் புருஷோத்தமம்}
{ஸ்துவன் - நாம ஸஹஸ்ரேண}{புருஷ: ஸததோத்\y{தி}{2}த: (ஸததோ / \y{தி}{2}த:)}

\newpage
% tameva cārcayannityaṁ bhaktyā puruṣamavyayam |
% dhyāyan stuvannamasyaṁśca yajamānastameva ca || 11 ||
\slokasl
{தம் - ஏவ (தமேவ) சார்சயன் - நித்யம்} {\y{ப}{4}க்த்யா புருஷம் - அவ்யயம்}
{\y{த்}{4}யாயன் ஸ்துவன் நமஸ்சன் - ஷ்ச} {யஜமான: தம்- ஏவ (யஜமானஸ்/தமேவ) ச}

% anādi nidhanaṁ viṣṇuṁ sarvalōkamaheśvaram |
% lōkādhyakṣaṁ stuvannityaṁ sarvaduḥkhātigō bhavet || 12 ||
\slokasl
{அனா\y{தி}{3} நி\y{த}{4}னம் விஷ்ணும்} {ஸர்வ லோக மஹேஷ்வரம்}
{லோகா - அ\y{த்}{4}யக்ஷம் ஸ்துவன் - நித்யம்} {சர்வ \y{து}{3}க்காதி\y{கொ}{3} \y{ப}{4}வேத்}

% brahmaṇyaṁ sarvadharmajñaṁ lōkānāṁ kīrtivardhanam |
% lōkanāthaṁ mahadbhūtaṁ sarvabhūtabhavōdhbhavam || 13 ||
\slokasl
{\y{ப்}{3}ரஹ்மண்யம் ஸர்வ \Y{த}{4}ர்மண்யம்} {\BU{லோ}கானாம் கீர்தி வர்\y{த}{4}னம்}
{லோகநா\y{த}{2}ம் ம\BU{ஹ\y{த்}{3}} - \y{பூ}{4}தம்} {\BU{ச}ர்வ \Y{பூ}{4}த \y{ப}{4}வோ\y{த்}{3}\y{ப}{4}வம்}

% eṣa me sarvadharmāṇāṁ dharmōdhikatamō mataḥ
% yadbhaktyā puṁḍarīkākṣaṁ stavairarcennaraḥ sadā || 14 ||
\slokasl
{ஏஶ மே ஸர்வ \Y{த}{4}ர்மாணாம்}{ \Y{த}{4}ர்மோ - அ\y{தி}{4}கதமோ மத:}
{ய\y{த்}{3} - \y{ப}{4}க்த்யா புண்\y{ட}{3}\BU{ரீ}காக்ஷம்}{\BU{ஸ்த}வை அர்சேன் நர: (ஸ்தவைர்/அர்சேன்னர:) ச\y{தா}{3}}

% paramaṁ yō mahattejaḥ paramaṁ yō mahattapaḥ |
% paramaṁ yō mahadbrahma paramaṁ yaḥ parāyaṇam || 15 ||
\slokasl
{பரமம் யோ மஹத்தேஜ:}{பரமம் யோ மஹத்தப:}
{பரமம் யோ மஹத்ப்ரஹ்ம}{பரமம் ய: பராயணம்}

\slokasl
{பவித்ராணாம் பவித்ரம் யோ}{மங்\y{க}{3}ளானாம் ச மங்\y{க}{3}ளம்}
{\y{தை}{3}ய்வதம் \y{தே}{3}வதானாம் ச}{\y{பூ}{4}தானாம் யொவ்யயஹ் பிதா}

% \phoneticsguide
} \end{document}
