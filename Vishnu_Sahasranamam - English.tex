\documentclass[10pt]{article}
\usepackage[utf8]{inputenc}

% Custom utility package to deal with Sanskrit accents in Tamil
\usepackage{stackmacros}

\begin{document}

\heading{\B{\Huge Vishnu Sahasranamam in English}}
\newcounter{mycounter}
% % śuklāṁbaradharaṁ viṣṇuṁ śaśivarṇaṁ caturbhujam |
% % prasannavadanaṁ dhyāyet sarvavighnōpaśāṁtaye || 1 ||


% \BU{\huge Poorva Peetika}\\

% % vyāsaṁ vasiṣṭhanaptāraṁ śakteḥ pautramakalmaṣam |
% % parāśarātmajaṁ vaṁde śukatātaṁ tapōnidhim || 3 ||

% % vyāsāya viṣṇurūpāya vyāsarūpāya viṣṇave |
% % namō vai brahmanidhaye vāsiṣṭhāya namō namaḥ || 4 ||

% \newpage

% % avikārāya śuddhāya nityāya paramātmane |
% % sadaikarūparūpāya viṣṇave sarvajiṣṇave || 5 ||

% % yasya smaraṇamātreṇa janmasaṁsārabaṁdhanāt |
% % vimucyate namastasmai viṣṇave prabhaviṣṇave || 6 ||

% % ōṁ namō viṣṇave prabhaviṣṇave ||

% % śrī vaiśaṁpāyana uvāca
% % śrutvā dharmānaśeṣeṇa pāvanāni ca sarvaśaḥ |
% % yudhiṣṭhiraḥ śāṁtanavaṁ punarevābhyabhyāṣata || 7 ||

% \newpage
% % śrī yudhiṣṭhira uvāca
% % kimekaṁ daivataṁ lōke kiṁ vāpyekaṁ parāyaṇaṁ |
% % stuvaṁtaḥ kaṁ kamarcaṁtaḥ prāpnuyurmānavāḥ śubham || 8 ||

% % kō dharmaḥ sarvadharmāṇāṁ bhavataḥ paramō mataḥ |
% % kiṁ japanmucyate jaṁturjanmasaṁsārabaṁdhanāt || 9 ||

% % śrī bhīṣma uvāca
% % jagatprabhuṁ devadevamanaṁtaṁ puruṣōttamam |
% % stuvannāmasahasreṇa puruṣaḥ satatōtthitaḥ || 10 ||

% tameva cārcayannityaṁ bhaktyā puruṣamavyayam |
% dhyāyan stuvannamasyaṁśca yajamānastameva ca || 11 ||
\slokas
{Tam-eva ch\y{aa}{---}rchayan - nityam}{B\x{h}{•}aktya Purusham - avyayam}
{D\x{h}{•}yayan stuvan namasyan-cha} {Yajam\y{aa}{---}nah tam-eva cha (Yajam\y{an}{---}as/tam-eva cha)}


% anādi nidhanaṁ viṣṇuṁ sarvalōkamaheśvaram
% lōkādhyakṣaṁ stuvannityaṁ sarvaduḥkhātigō bhavet || 12 ||
\slokas{An\y{aa}{---}di ni\x{dh}{•}anam vishnum}{Sarva loka mah\y{e}{--}shwaram}
{Loka - a\x{dh}{•}yaksham Stuvan - nithyam}{Sarva du\x{kh}{•}\y{a}{--}tigo B\x{h}{•}aveth}

% brahmaṇyaṁ sarvadharmajñaṁ lōkānāṁ kīrtivardhanam |
% lōkanāthaṁ mahadbhūtaṁ sarvabhūtabhavōdhbhavam || 13 ||

% eṣa me sarvadharmāṇāṁ dharmōdhikatamō mataḥ
% yadbhaktyā puṁḍarīkākṣaṁ stavairarcennaraḥ sadā || 14 ||

% paramaṁ yō mahattejaḥ paramaṁ yō mahattapaḥ |
% paramaṁ yō mahadbrahma paramaṁ yaḥ parāyaṇam || 15 ||

\end{document}
