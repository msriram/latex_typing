\documentclass[20pt]{article}
\usepackage[utf8]{inputenc}

% Custom utility package to deal with Sanskrit accents in Tamil
\usepackage{stackmacros}
\usepackage{fontspec}
\defaultfontfeatures{Mapping=tex-text,Scale=MatchLowercase}
\setmainfont{Arial}
\setmonofont{Lucida Sans Typewriter}
\begin{document}

\heading{\BU{\Huge VISHNU SAHASRANAMA}}
\newcounter{mycounter}

% śuklāṁbaradharaṁ viṣṇuṁ śaśivarṇaṁ caturbhujam |
% prasannavadanaṁ dhyāyet sarvavighnōpaśāṁtaye || 1 ||
\slokash{Shukla ambaradaram vishnum} {Shashi varnam chatur bhujam}
{Prasanna vadanam dhyayet} {Sarva vignopa shantaye  }

\BU{\huge POORVA PEETIKA }

% vyāsaṁ vasiṣṭhanaptāraṁ śakteḥ pautramakalmaṣam |
% parāśarātmajaṁ vaṁde śukatātaṁ tapōnidhim || 3 ||
\slokash{Vyasam vasishta naptharam } {Shakteh poutram - akalmasham }
{Parashara - atmajam vande} {Shuka thatham tapo nidhim  }

% vyāsāya viṣṇurūpāya vyāsarūpāya viṣṇave |
% namō vai brahmanidhaye vāsiṣṭhāya namō namaḥ || 4 ||
\slokash{Vyasaya vishnu roopaya } {Vyasa roopaya vishnave } 
{Namo vai brahma nidhaye } {Vaasishtaya namo namaha }
% \newpage

% avikārāya śuddhāya nityāya paramātmane |
% sadaikarūparūpāya viṣṇave sarvajiṣṇave || 5 ||
\slokash{Avikaraya shuddhaya } {Nithyaya paramatmane } 
{Sadaika roopa roopaya } {Vishnave sarva jishnnave  }

% yasya smaraṇamātreṇa janmasaṁsārabaṁdhanāt |
% vimucyate namastasmai viṣṇave prabhaviṣṇave || 6 ||
\slokash{Yasya smarana matrena } {Janma samsara bandhanath } 
{ Vimuchyate namah - tasmai (namas / tasmai ) } {Vishnave prabha vishnave }

% ōṁ namō viṣṇave prabhaviṣṇave || }
\slokah{Aum namo vishnave prabha vishnave}

% śrī vaiśaṁpāyana uvāca
% śrutvā dharmānaśeṣeṇa pāvanāni ca sarvaśaḥ |
% yudhiṣṭhiraḥ śāṁtanavaṁ punarevābhyabhyāṣata || 7 ||
\slokadh{Sri vaishampayana uvacha }
{Shruthva dharman - asheshena }{Pavanani cha sarvashaha }
{Yudhishtirah shantanavam }{Punareva - abhi - abhashata ( abhya / bhashatha )}

% \newpage

% śrī yudhiṣṭhira uvāca
% kimekaṁ daivataṁ lōke kiṁ vāpyekaṁ parāyaṇaṁ |
% stuvaṁtaḥ kaṁ kamarcaṁtaḥ prāpnuyurmānavāḥ śubham || 8 ||
\slokadh {Yudhishtira uvacha }
{Kim - ekam daivatam loke } {Kim vapi - ekam ( vapye / kam ) parayanam}
{Stuvantah kam kam - archantah } {Prapnuyuh - manavah ( Prapnuyur / manavah ) shubham }

% kō dharmaḥ sarvadharmāṇāṁ bhavataḥ paramō mataḥ | }
% kiṁ japanmucyate jaṁturjanmasaṁsārabaṁdhanāt || 9 ||
\slokash{Ko dharmah sarva dharmanaam } {Bhavatah paramo mataha } 
{Kim japan - muchyate jantuhu } {Janma samsara bandhanath ( jantur / janma )  }

% śrī bhīṣma uvāca
% jagatprabhuṁ devadevamanaṁtaṁ puruṣōttamam |
% stuvannāmasahasreṇa puruṣaḥ satatōtthitaḥ || 10 ||
\slokadh{Sri Bheeshma uvacha}
{Jagat prabhum deva devam } {Anantham purushottamam } 
{Stuvan - nama sahasrena } {Purushaha satatothitaha ( Satato / thitaha ) }

% tameva cārcayannityaṁ bhaktyā puruṣamavyayam |
% dhyāyan stuvannamasyaṁśca yajamānastameva ca || 11 ||
\slokash
{Tam-eva ch\y{aa}{---}rchayan - nityam}{B\x{h}{•}aktya Purusham - avyayam}
{D\x{h}{•}y\y{a}{--}yan stuvan namasyan-cha} {Yajam\y{aa}{---}nah tam-eva cha (Yajam\y{an}{---}as/tam-eva cha)}


% anādi nidhanaṁ viṣṇuṁ sarvalōkamaheśvaram
% lōkādhyakṣaṁ stuvannityaṁ sarvaduḥkhātigō bhavet || 12 ||
\slokash{An\y{aa}{---}di ni\x{dh}{•}anam vishnum}{Sarva loka mah\y{e}{--}shwaram}
{Loka - a\x{dh}{•}yaksham Stuvan - nityam}{Sarva du\x{kh}{•}\y{a}{--}tigo B\x{h}{•}aveth}

% brahmaṇyaṁ sarvadharmajñaṁ lōkānāṁ kīrtivardhanam |
% lōkanāthaṁ mahadbhūtaṁ sarvabhūtabhavōdhbhavam || 13 ||
\slokash{Brahmanyam sarva \x{Dh}{•}armangnam} {L\y{o}{--}k\y{a}{--}n\y{a}{--}m K\y{ee}{---}rthi Var\x{dh}{•}anam}
{L\y{o}{--}kan\y{a}{--}tham Mahad - \x{Bh}{•}\y{oo}{---}tam} {Sarva Bh\y{oo}{---}tha \x{Bh}{•}av\y{o}{--}d\x{bh}{•}avam}

% eṣa me sarvadharmāṇāṁ dharmōdhikatamō mataḥ
% yadbhaktyā puṁḍarīkākṣaṁ stavairarcennaraḥ sadā || 14 ||
\slokash{Esha me sarva \x{Dh}{•}arm\y{a}{--}n\y{a}{--}m}{\x{Dh}{•}ar\y{mo}{---} - A\x{dh}{•}ikata\y{mo}{---} mataha}
{Yat - \x{Bh}{•}akty\y{a}{--} Pundareek\y{a}{--}ksham} {Stavaih arch\y{e}{--}n narah sad\y{a}{--} (Stavair/arch\y{e}{--}n/narah)}
% paramaṁ yō mahattejaḥ paramaṁ yō mahattapaḥ |
% paramaṁ yō mahadbrahma paramaṁ yaḥ parāyaṇam || 15 ||

\end{document}
